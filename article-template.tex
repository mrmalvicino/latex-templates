\documentclass[a4paper,12pt]{article}
% \setlength{\headheight}{15pt} % Sirve para corregir un error asociado a fancyhdr

\input{article-packages}
\input{article-commands}

\begin{document}

\begin{center}
    % \includegraphics[width=8cm]{Logo Universidad.png}
    
    \vspace{5cm} % Achicar carátula con esta medida
    \begin{Large}
        Nombre de Carrera.
        \\ \vspace{1cm}
        Nombre de Asignatura 1.
    \end{Large}
    \vspace{2cm}

    \begin{Huge}
        \textbf{Título del informe.}
    \end{Huge}

    \vspace{1cm}
    \textbf{dd/mm/aa}
    \vspace{5cm} % Achicar carátula con esta medida

    \begin{Large}
        Malvicino, Maximiliano R. \\ mrmalvicino@gmail.com
        % \\ \vspace{1cm}
        % Apellido, Nombre. \\ mail@gmail.com
        % \\ \vspace{1cm}
        % Apellido, Nombre. \\ mail@gmail.com
        % \\ \vspace{1cm}
        
    \end{Large}

\end{center}


\clearpage

\begin{abstract}
    El objetivo de este trabajo es ...
    
    % \begin{center}
    %     \textbf{Abstract}
    % \end{center}

    % The purpose of this project is ...

\end{abstract}

\clearpage

% \renewcommand{\spanishappendixname}{Anexo}
% \renewcommand{\contentsname}{Índice global}
\tableofcontents

% \renewcommand\listfigurename{}
% \listoffigures

% \renewcommand\spanishtablename{Tabla}
% \renewcommand\listtablename{Índice de tablas}
% \listoftables


\clearpage


\begin{multicols}{2}


% INTRODUCCIÓN
\section{Introducción}


% MARCO TEÓRICO
\section{Marco teórico}


% DESARROLLO EXPERIMENTAL
\section{Desarrollo experimental}


\subsection{Lista de materiales usados}

\begin{itemize}
    \item Material 1
    \item Material 2
    \item Material 3
    \item Material n
\end{itemize}


% RESULTADOS
\section{Resultados}


% CONCLUSIONES
\section{Conclusiones}


\end{multicols}

\clearpage
\begin{thebibliography}{1}
    \bibitem{1} \textsc{Apellido, N. (Año). Título.}
    % \url{https://www.google.com}
\end{thebibliography}


% \clearpage
% \appendix


% \section{Anexo}


\end{document}